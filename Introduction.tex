%%%%%%%%%%%%%%%%%%%%%%%%%%%%%%%%%%%%%%%%%%%%%%%%%%%%%%%%%%%%%%%%%%%%%%
%     File: ExtendedAbstract_intro.tex                               %
%     Tex Master: ExtendedAbstract.tex                               %
%                                                                    %
%     Author: Andre Calado Marta                                     %
%     Last modified : 27 Dez 2011                                    %
%%%%%%%%%%%%%%%%%%%%%%%%%%%%%%%%%%%%%%%%%%%%%%%%%%%%%%%%%%%%%%%%%%%%%%
% State the objectives of the work and provide an adequate background,
% avoiding a detailed literature survey or a summary of the results.
%%%%%%%%%%%%%%%%%%%%%%%%%%%%%%%%%%%%%%%%%%%%%%%%%%%%%%%%%%%%%%%%%%%%%%

\section{Introduction}
\label{Introduction}

Asset management is a crucial part of any museum. For a museum manager it is imperative to have the capability of inserting, viewing, editing and removing any asset at any given moment.


In the past, the main way of managing these assets was through paper registries, but with the emergence and development of information technologies we are now able to perform these task more efficiently and without as big of a risk of losing the information.

Despite the technological progress, there still isn't an ideal way of dealing with museological assets mainly due to their complexity and diversity. There exists already some software for performing these tasks but they are generally closed in the sense that the client is unable to adapt the platform to fit its specific needs.


The motivation for the development of this work comes with the anaylsis of a particular asset management case, more concretely, the one at Instituto Superior Técnico (IST). IST is a portuguese engineering college with a wide range of museological assets to manage, despite the lack of an estabilished system required to perform these necessary tasks.


As previously mentioned, one of the biggest challenges comes with the diversity of assets. IST is a perfect example of this scenario, having assets that range from rock and minerals all the way to motors and electronic devices. Another challenge comes with the lack of procedures which are currently not defined for IST.


Also, having an immediate need for such a management system, IST is the ideal case study to focus this work upon.


There already exist other systems in the market, such as closed-code products developed and sold by private companies, as well as opensourced solutions developed by the community. But still, with all the available solutions, none of them gives the necessary response to an institution such as IST.
