%%%%%%%%%%%%%%%%%%%%%%%%%%%%%%%%%%%%%%%%%%%%%%%%%%%%%%%%%%%%%%%%%%%%%%
%     File: ExtendedAbstract_abstr.tex                               %
%     Tex Master: ExtendedAbstract.tex                               %
%                                                                    %
%     Author: Andre Calado Marta                                     %
%     Last modified : 2 Dez 2011                                     %
%%%%%%%%%%%%%%%%%%%%%%%%%%%%%%%%%%%%%%%%%%%%%%%%%%%%%%%%%%%%%%%%%%%%%%
% The abstract of should have less than 500 words.
% The keywords should be typed here (three to five keywords).
%%%%%%%%%%%%%%%%%%%%%%%%%%%%%%%%%%%%%%%%%%%%%%%%%%%%%%%%%%%%%%%%%%%%%%

%%
%% Abstract
%%
\begin{abstract}

Asset management in museums has seen continuous improvements. With the advancements made in technology collection museums have evolved from using paper based registries to using digital systems that facilitate a collection manager's life as well the efficiency of performing modification to the asset base and registering these modifications.

Despite the currently available systems for asset management within museums, there is not a well defined best solution since these assets contain very diverse properties and requirements when cataloguing and storing their information.

The motivation for this work comes from the desire and necessity of creating a single highly adaptable and scalable solution capable of giving response to most museums.

This work has two main objectives: provide a long-term term solution that can be adapted for all museums to use in order to deal with their requirements when managing their collection of assets, and a short-term solution of providing a system capable of serving as a tool to solve IST's immediate need to catalogue its collection of assets from its multiple collections.

The motivation for the long-term objective comes with the current state of asset management solution, with most platforms providing a closed solution which is not meant to be adapted.

For the short-term objective, we are faced with IST, a university with a vast collection of assets from multiple areas which are not currently catalogued.

This work also provides insights into some of the currently used procedures when it comes to asset management within museums and how they differ from each other.
\\
%%
%% Keywords (max 5)
%%
\noindent{{\bf Keywords:}} Museum; Database; Asset Management; Extensibility; Web; Vue.js; Node.js; MongoDB \\

\end{abstract}

