%%%%%%%%%%%%%%%%%%%%%%%%%%%%%%%%%%%%%%%%%%%%%%%%%%%%%%%%%%%%%%%%%%%%%%
%     File: ExtendedAbstract_backg.tex                               %
%     Tex Master: ExtendedAbstract.tex                               %
%                                                                    %
%     Author: Andre Calado Marta                                     %
%     Last modified : 27 Dez 2011                                    %
%%%%%%%%%%%%%%%%%%%%%%%%%%%%%%%%%%%%%%%%%%%%%%%%%%%%%%%%%%%%%%%%%%%%%%
% A Theory section should extend, not repeat, the background to the
% article already dealt with in the Introduction and lay the
% foundation for further work.
%%%%%%%%%%%%%%%%%%%%%%%%%%%%%%%%%%%%%%%%%%%%%%%%%%%%%%%%%%%%%%%%%%%%%%

\section{Background}

One of the difficulties of cataloguing the assets of museums comes with the vast amount of specific locations an asset can be in. Museums such as the museum of Geology and Mineralogy have most of their assets within cabinets, drawers and sections. This requires the documentation of the asset to be very precise in stating its location.

Another difficulty is related to the amount o properties associated to assets of different categories. The properties used to catalog and identify a painting and a mineral will be very different.

The thesis \cite{Barros2007MaterialidadeIST} developed by Miriam resulted on a Microsoft Access document with information on some of the assets of the museum of Geology and Mineralogy at IST. The document includes 3 different tables: identification, characteristics and history. These are the fields used to describe assets of this museum:

\vspace{4mm}
\textbf{Identification}
\begin{multicols}{2}
\begin{itemize}
    \item Inventory number
    \item Inventory A
    \item Inventory B
    \item Inventory C
    \item Designation 1
    \item Designation 2
    \item Designation 3
    \item Brand
    \item Collection
    \item Location
\end{itemize}
\end{multicols}
\vspace{4mm}
\textbf{Characteristics}
\begin{multicols}{2}
\begin{itemize}
    \item Inventory number
    \item Height
    \item Depth
    \item Diameter
    \item Weight
    \item Wood material
    \item Glass material
    \item Metal material
    \item Plastic material
    \item Other material
    \item Price
    \item Price assigner
\end{itemize}
\end{multicols}


\textbf{History}
\begin{multicols}{2}
\begin{itemize}
    \item Inventory number
    \item Acquisition date
    \item Acquisition method
    \item Acquisition price
    \item Supplier
    \item Conservation
    \item Repair
    \item Bibliography
    \item Exhibitions
    \item Sites 1
    \item Sites 2
    \item Sites 3
    \item Sites 4
    \item Documentation
\end{itemize}
\end{multicols}

\vspace{20mm}
\textbf{Spectrum}
\vspace{2mm}

%https://collectionstrust.org.uk/spectrum/spectrum-5/
Spectrum\cite{CollectionsTrust2009TheProcedure} is a collection management standard developed in the UK that is also used in other places around the world with the latest version having been published in September of 2017. According to its creators, Spectrum gives "tried-and-tested advice on the things most museums do when managing their collections".


This standard is activity-oriented, it focuses on how to register and execute certain activities, such as moving an object or updating location records. These activities are called procedures and Spectrum has a total of 21 defined procedures.


Each procedure has the following elements:

\begin{itemize}
    \item A definition that sums up the procedure into a single sentence.
    \item An explanation on the scope, explaining when the procedure should be used.
    \item The spectrum standard, that consists on the goal that we plan on achieving.
    \item A suggested workflow, this can be a diagram or a text written explanation.
\end{itemize}

When it comes to concrete procedures, these are the ones that Spectrum \cite{CollectionsTrust2009TheProcedure} considers to be its primary procedures:

\begin{itemize}
    \item Object entry - Registering incoming assets taking in to account the entry method (acquisition, loan, ...).
    \item Acquisition and accessioning - Taking legal ownership of objects in order to add them into the museum's collection.
    \item Location and movement control - Keeping track of the locations of all assets and updating them each time the assets are moved.
    \item Inventory - Making sure you have the basic information in order to identify your objects.
    \item Cataloguing - Managing the assets' information.
    \item Object exit - Recording when, the assets you are responsible for, exit your collection.
    \item Loans in - Managing objects you borrow for a fixed period of time and for a specific purpose.
    \item Loans out - Managing the requests for you to lend assets as well as the lending process until assets are returned to you.
    \item Documentation planning - Ongoing improvement of your documentation systems.
\end{itemize}

Taking this into account, Spectrum is not an all one single document but the composition of multiple documents, being that each one of these is associated to a specific procedure.


Spectrum is not a software itself but a set of rules that you should follow in order to create a digital of paper system that complies this management standard. There is not a single way of implementing this standard as we can see in the documents associated to the procedures, but as long as the base rules are followed we can consider a system to be compliant with this standard.